% Created 2023-08-29 火 17:12
% Intended LaTeX compiler: pdflatex
\documentclass[10pt]{article}
\usepackage[utf8]{inputenc}
\usepackage[T1]{fontenc}
\usepackage{graphicx}
\usepackage{longtable}
\usepackage{wrapfig}
\usepackage{rotating}
\usepackage[normalem]{ulem}
\usepackage{amsmath}
\usepackage{amssymb}
\usepackage{capt-of}
\usepackage{hyperref}
\usepackage[newfloat]{minted}
\usepackage[a4paper, total={6.5in, 9in}]{geometry}
\usepackage{minted}
\setminted{breaklines}
\usepackage[utf8]{inputenc}
\renewcommand{\familydefault}{\sfdefault}
\usemintedstyle{vs}
\usepackage[most]{tcolorbox}
\usepackage{CJKutf8}
\usepackage{xurl}
\usepackage{fontawesome5}
\usepackage{hyperref}
\usepackage{graphicx}
\usepackage{float}
\newcommand{\gitlab}[1]{%
\href{#1}{GitLab \faGitlab}}
\author{Vincent Conus}
\date{2023-8-24}
\title{Setting up and using Xilinx KRIA KV260\\\medskip
\large \begin{CJK}{UTF8}{min}南山大学\end{CJK}}
\hypersetup{
 pdfauthor={Vincent Conus},
 pdftitle={Setting up and using Xilinx KRIA KV260},
 pdfkeywords={},
 pdfsubject={A report presenting how to use and set Xilinx's Kria board},
 pdfcreator={Emacs 30.0.50 (Org mode 9.6.6)}, 
 pdflang={English}}
\begin{document}

\begin{titlepage}
\centering
{\LARGE Setting up and using Xilinx KRIA KV260 \par }
\vspace{5mm}
{\large \begin{CJK}{UTF8}{min}南山大学\end{CJK} \par}
\vspace{1cm}
{\large 2023-8-24 \par}
\vspace{2cm}
{\large Vincent Conus -  Source available at \gitlab{https://gitlab.com/sunoc/xilinx-kria-kv260-documentation} \par}
\vspace{3cm}
\includegraphics[width=0.8\textwidth]{./img/board}\end{titlepage}
\tableofcontents
\pagebreak
\section{Introduction \& motivation}
\label{sec:org363fbb7}
This guide will present how to setup and use Xilinx's KRIA board, in particular
for running ROS on a host Ubuntu system, as well as for deploying
micro-ROS as a firmware on the MCU part of this board's chip.

The use of this device in particular is interesting because of the presence
of a CPU comprising both a general purpose ARM core, capable of running
a Linux distribution, as well as another ARM core, real-time enabled,
capable to run a RTOS.

\section{Boot firmware}
\label{sec:org54915a9}
The goal for the Linux side of the deployment is to
have the latest LTS version of Ubuntu up and running.
In order to be able to boot such a newer version of Linux, the
boot image of the board must first be updated.

The procedure is available in \href{https://docs.xilinx.com/r/en-US/ug1089-kv260-starter-kit/Firmware-Update}{the official documentation},
but I will present it step by step here.

\subsection{Getting the new firmware}
\label{sec:org1c03638}
A 2022 version of the board firmware is required in order to run the latest
version of Ubuntu properly.

The image can be downloaded at \href{https://xilinx-wiki.atlassian.net/wiki/spaces/A/pages/1641152513/Kria+K26+ SOMoot-FW-update-with-xmutil}{the atlassian page} on the topic,
or even directly with the following command:

\begin{minted}[frame=single,framesep=2mm,baselinestretch=1.2,linenos,breaklines,fontsize=\footnotesize]{sh}
wget https://www.xilinx.com/member/forms/download/\
     design-license-xef.html?filename=BOOT-k26-starter-kit-20230516185703.bin
\end{minted}


\subsection{Reaching the board recovery tool}
\label{sec:org5a15ab0}
Now the firmware \texttt{.bin} image is available, it is possible to update it using the
boards recovery tool. Here are the steps that must be taken in order to reach
this tool and update the board:

\begin{itemize}
\item Connect the board to your machine via a Ethernet cable.
This will obviously cut you internet access, so you should be set for that.
\item Select the wired network as your connection (must be "forced", since it
doesn't have internet access).
\item Set a fixed IP address for your machine, in the \texttt{192.168.0.1/24}
range, except the specific \texttt{192.168.0.111}, which will be used by the
board.
\item Using a web browser on your host machine, access
\texttt{http://192.168.0.111}. Thou shall now see the interface, as visible on
the figure \ref{fig:orgd39eadd} below.
\end{itemize}

\begin{figure}[htbp]
\centering
\includegraphics[width=1\textwidth]{img/recovery.png}
\caption{\label{fig:orgd39eadd}The recovery tool for the board, access from Firefox. We can see board information at the center, and the tools to upload the firmware at the bottom of the page.}
\end{figure}

\subsection{Updating the boot firmware}
\label{sec:org4ae6ff4}
From this "recovery" page, it is possible to upload the \texttt{.bin} file downloaded previously onto
the board using the "Recover Image" section at the bottom right of the page.

The board can be re-booted afterwards.

\section{Installing Linux}
\label{sec:org1ccbc9e}
Withe the boot firmware being up-to-date, we can proceed to install a Linux distribution
on our Kria board. The step needed to archive a full installation of Ubuntu 22.04
will be presented in this section.

\subsection{Preparing and booting a Ubuntu 22.04 media}
\label{sec:org0c0e148}
An \href{https://ubuntu.com/download/amd-xilinx}{official Ubuntu image} exists and is
provided by Xilinx, allowing the OS installation to be quick and
straightforward.
Ubuntu is a common and easy to use distribution. Furthermore,
it allows to install ROS2 as a package, which is most convenient and will be
done later in this guide.

Once the image has been downloaded at \href{https://ubuntu.com/download/amd-xilinx}{Canonical's page}
we can flash it onto the SD card, with the following instructions.

\begin{tcolorbox}[colback=red!5!white,colframe=red!75!black]
\textbf{DANGER}: The next part involve the \texttt{dd} command writing on disks!!!
As always with the dd command, thou have to be \textbf{VERY} careful on what arguments
thou give. Selecting the wrong disk will result on the destruction of
thy data !!
\uline{If you are unsure of what to do, seek assistance !}
\end{tcolorbox}

With the image available on thy machine and a SD card visible as \texttt{/dev/sda}
Once the SD card is flashed and put back in the board, the micro-USB cable can be
connected from the PC to the board. It is then possible to
connect to the board in serial with an appropriate tool, for example \texttt{picocom},
as in the following example (the serial port that "appeared" was the \texttt{/dev/ttyUSB1} in this case,
and the 115200 bitrate is the default value for the board):

\begin{minted}[frame=single,framesep=2mm,baselinestretch=1.2,linenos,breaklines,fontsize=\footnotesize]{sh}
sudo picocom /dev/ttyUSB1 -b 115200
\end{minted}

Once logged in, it is typically easier and more convenient to connect the board
using SSH. When the board is connected to the network, it is possible to know
it's IP address with the \texttt{ip} command; then it is possible to connect to
the board with ssh, as follow (example, with the first command to be run on the board
and the second one on the host PC, both without the first placeholder hostnames):


\begin{minted}[frame=single,framesep=2mm,baselinestretch=1.2,linenos,breaklines,fontsize=\footnotesize]{sh}
kria# ip addr

host# ssh ubuntu@192.168.4.11
\end{minted}

\subsection{Network and admin setups}
\label{sec:org0e04853}
This section presents a variety of extra convenience configurations
that can be used when setting-up the Kria board.

\subsubsection{Proxy and DNS}
\label{sec:orge7a7828}
An issue that can occur when connecting the board to the internet is the
conflicting situation with the university proxy.
Indeed, as the network at Nanzan University requires to go through a proxy,
some DNS errors appeared.

Firstly, it is possible to set a DNS IP address in \texttt{/etc/resolv.conf} by
editing it and adding your favorite DNS, for example \texttt{nameserver 1.1.1.1}
next to the other \texttt{nameserver} entry. The resolver can then be restarted.

\begin{minted}[frame=single,framesep=2mm,baselinestretch=1.2,linenos,breaklines,fontsize=\footnotesize]{sh}
sudo nano /etc/resolv.conf

sudo systemctl restart systemd-resolved
\end{minted}

Secondly, it might become needed to setup the proxy for the school.

This can be done as follow, by exporting a https base proxy configuration
containing you AXIA credentials (this is specific to Nanzan University IT system),
then by consolidating the configuration for other types of connections in the \texttt{bashrc}:

\begin{minted}[frame=single,framesep=2mm,baselinestretch=1.2,linenos,breaklines,fontsize=\footnotesize]{sh}
export https_proxy="http://<AXIA_username>:\
       <AXIA_psw>@proxy.ic.nanzan-u.ac.jp:8080"

echo "export http_proxy=\""$https_proxy"\"" >> ~/.bashrc \
     echo "export https_proxy=\""$https_proxy"\"" >> ~/.bashrc \
     echo "export ftp_proxy=\""$https_proxy"\"" >> ~/.bashrc \
     echo "export no_proxy=\"localhost, 127.0.0.1,::1\"" \
     >> ~/.bashrc
\end{minted}

Eventually the board can be rebooted in order for the setup to get applied cleanly.

\subsubsection{\texttt{root} password}
\label{sec:org7f641d8}
\begin{tcolorbox}[colback=orange!5!white,colframe=orange!75!black]
\textbf{WARNING}: Depending on your use-case, the setup presented in this
subsection can be a critical security breach as it remove the need for a root
password to access the admin functions of the board's Linux.
\uline{When in doubt, do not apply this configuration!!}
\end{tcolorbox}

If you board does not hold important data
and is available to you only, for test or development,
it might be convenient for the \texttt{sudo} tool to not ask for the
password all the time.
This change can be done by editing the sudoers file, and
adding the parameter \texttt{NOPASSWD}
at the \texttt{sudo} line:

\begin{minted}[frame=single,framesep=2mm,baselinestretch=1.2,linenos,breaklines,fontsize=\footnotesize]{sh}
sudo visudo

%sudo   ALL=(ALL:ALL) NOPASSWD: ALL
\end{minted}

Again, this is merely a convenience setup for devices staying at you desk. If
the board is meant to be used in any kind of production setup, a password
should be set for making administration tasks.

With all of these settings, you should be able to update the software of your
board without any issues:
\begin{minted}[frame=single,framesep=2mm,baselinestretch=1.2,linenos,breaklines,fontsize=\footnotesize]{sh}
sudo apt-get update
sudo apt-get dist-upgrade
sudo reboot now
\end{minted}


\subsubsection{Static IP address}
\label{sec:orgfb72bb0}
A static IP can be set by writing the following
configuration into your \texttt{netplan} configuration file.

The name of the files might vary:
\begin{minted}[frame=single,framesep=2mm,baselinestretch=1.2,linenos,breaklines,fontsize=\footnotesize]{sh}
sudo nano /etc/netplan/50-cloud-init.yaml
\end{minted}

You can then set the wanted IP as follow. Note that a custom DNS was
also set in that case.
\begin{minted}[frame=single,framesep=2mm,baselinestretch=1.2,linenos,breaklines,fontsize=\footnotesize]{yaml}
network:
  renderer: NetworkManager
  version: 2
  ethernets:
    eth0:
      addresses:
        - 192.168.11.103/24
      routes:
        - to: default
          via: 192.168.11.1
      nameservers:
        addresses:
          - 8.8.8.8
          - 1.1.1.1
\end{minted}

Finally, the change in settings can be applied
as follow:

\begin{minted}[frame=single,framesep=2mm,baselinestretch=1.2,linenos,breaklines,fontsize=\footnotesize]{sh}
sudo netplan apply
\end{minted}

\subsubsection{Purging \texttt{snap}}
\label{sec:org954490a}
As the desktop-specific software are not used at all in the case
of our project, there are some packages that can be purges in order for the
system to become more lightweight.

In particular, the main issue with Ubuntu systems is the forced integration of
Snap packages. Here are the command to use in order to remove all of that.
These steps take a lot of time and need to be executed in that specific order\footnote{The \texttt{snap} package depends on each other. Thus dependencies
cannot be remove before the package(s) that depends on them.},
but the system fan runs sensibly slower without all of this stuff:

\begin{minted}[frame=single,framesep=2mm,baselinestretch=1.2,linenos,breaklines,fontsize=\footnotesize]{sh}
sudo systemctl disable snapd.service
sudo systemctl disable snapd.socket
sudo systemctl disable snapd.seeded.service

sudo snap list #show installed package, remove then all:
sudo snap remove --purge firefox
sudo snap remove --purge gnome-3-38-2004
sudo snap remove --purge gnome-42-2204
sudo snap remove --purge gtk-common-themes
sudo snap remove --purge snapd-desktop-integration
sudo snap remove --purge snap-store
sudo snap remove --purge bare
sudo snap remove --purge core20
sudo snap remove --purge core22
sudo snap remove --purge snapd
sudo snap list # check that everything is uninstalled

sudo rm -rf /var/cache/snapd/
sudo rm -rf ~/snap
sudo apt autoremove --purge snapd

systemctl list-units | grep snapd
\end{minted}

\subsubsection{Other unused heavy packages}
\label{sec:org664fe94}
Some other pieces of software can safely be removed since the desktop is
not to be used:

\begin{minted}[frame=single,framesep=2mm,baselinestretch=1.2,linenos,breaklines,fontsize=\footnotesize]{sh}
sudo apt-get autoremove --purge yaru-theme-icon \
fonts-noto-cjk yaru-theme-gtk vim-runtime \
ubuntu-wallpapers-jammy humanity-icon-theme

sudo apt-get autoclean
sudo reboot now
\end{minted}

\section{Enabling \texttt{remoteproc}}
\label{sec:orgd814eab}
One of the advantage of this Kria board, as cited previously, is the presence of
multiple types of core (APU, MCU, FPGA) on the same chip.

The part in focus in this guide is the usage of both the APU, running
a Linux distribution and ROS2; and the MCU, running FreeRTOS and micro-ROS.
Online available guides\footnote{A \href{https://speakerdeck.com/fixstars/fpga-seminar-12-fixstars-corporation-20220727}{slideshow} (JP) from Fixstar employees presents how to use the device
tree to enable the communication between the cores.} \textsuperscript{,}\,\footnote{A \href{https://zenn.dev/ryuz88/articles/kv260\_setup\_memo\_ubuntu22 }{blog post} (JP) shows all major steps on how to enable the \texttt{remoteproc}.} also provide information on how to deploy these types
of systems and enabling \texttt{remoteproc} for the Kria board, but this guide
will show a step-by-step, tried process to have a heterogeneous system
up and running.

The communication between both side is meant to be done using shared memory, but
some extra setup is required in order to be running the real-time firmware, in particular
for deploying micro-ROS on it.

As a first step in that direction, this section of the report
will present how to setup and use as an example firmware that utilizes the
\texttt{remoteproc} device in Linux in order to access shared memory
and communicate with the real-time firmware using the RPMsg system.

\subsection{Device-Tree Overlay patching}
\label{sec:orge2e9001}
The communication system and interaction from the Linux side towards the real-time capable core
is not enabled by default within the Ubuntu image provided by Xilinx.

In that regard, some modification of the device tree overlay (DTO) is required in order to have
the \texttt{remoteproc} system starting.

Firstly, we need to get the original firmware device tree, converted
into a readable format (DTS):

\begin{minted}[frame=single,framesep=2mm,baselinestretch=1.2,linenos,breaklines,fontsize=\footnotesize]{sh}
sudo dtc /sys/firmware/fdt 2> /dev/null > system.dts
\end{minted}

Then, a custom-made patch file can be downloaded and applied.
This file is available at the URL visible in the command below
but also in this report appendix \ref{sec:orgf1db9a6}.

\begin{minted}[frame=single,framesep=2mm,baselinestretch=1.2,linenos,breaklines,fontsize=\footnotesize]{sh}
wget https://gitlab.com/sunoc/xilinx-kria-kv260-documentation/-/\
     blob/b7300116e153f4b5a1542f8804e4646db8030033/src/system.patch

patch system.dts < system.patch
\end{minted}

As for the board to be able to reserve the correct amount of memory with the new settings, some
\texttt{cma} kernel configuration is needed\footnote{The overlapping memory will not prevent the board to boot,
but it disables the PWM for the CPU fan, which will then run at full speed, making noise.}:

\begin{minted}[frame=single,framesep=2mm,baselinestretch=1.2,linenos,breaklines,fontsize=\footnotesize]{sh}
sudo nano /etc/default/flash-kernel

LINUX_KERNEL_CMDLINE="quiet splash cma=512M cpuidle.off=1"
LINUX_KERNEL_CMDLINE_DEFAULTS=""
sudo flash-kernel
\end{minted}

Now the DTS file has been modified, one can regenerate the binary and place it on the \texttt{/boot} partition
and reboot the board:

\begin{minted}[frame=single,framesep=2mm,baselinestretch=1.2,linenos,breaklines,fontsize=\footnotesize]{sh}
dtc -I dts -O dtb system.dts -o user-override.dtb
sudo mv user-override.dtb /boot/firmware/
sudo reboot now
\end{minted}

After rebooting, you can check the content of the \verb|remoteproc| system directory,
and a \texttt{remoteproc0} device should be visible, as follow:

\begin{minted}[frame=single,framesep=2mm,baselinestretch=1.2,linenos,breaklines,fontsize=\footnotesize]{sh}
ls /sys/class/remoteproc/
#  remoteproc0
\end{minted}

If it is the case, it means that the patch was successful and  that the remote processor is
ready to be used!

\section{Building an example RPMsg real-time firmware}
\label{sec:orgb786896}
As visible on the official \href{https://xilinx-wiki.atlassian.net/wiki/spaces/A/pages/1837006921/OpenAMP+Base+Hardware+Configurations\\\#Build-RPU-firmware}{Xilinx documentation about building a demo firmware},
this section will present the required steps for building a new firmware for the R5F
core of our Kria board.

The goal here is to have a demonstration firmware running,
able to use the RPMsg system to communicate with the Linux APU.

\subsection{Setting up the IDE}
\label{sec:orgf0c28dd}
Xilinx's Vitis IDE is the recommended tool used to build software for the Xilinx boards.
It also include the tools to interact with the FPGA part, making the whole
software very large (around 200GB of disk usage).

However, this large tool-set allows for a convenient development environment, in particular
in our case where some FreeRTOS system, with many dependencies is to be build.

The installer can be found on \href{https://www.xilinx.com/support/download/index.html/content/xilinx/en/downloadNav/vitis.html}{Xilinx download page}. You will need to get
a file named something like \texttt{Xilinx\_Unified\_2022.2\_1014\_8888\_Lin64.bin}\footnote{The name of the installer binary file might change as a new version of the IDE
is release every year or so.}.

Vitis IDE installer is compatible with versions of Ubuntu, among other distributions,
but not officially yet for the 22.04 version.
Furthermore, the current install was tested on Pop OS, a distribution derived from Ubuntu.
However, even with this more unstable status, no major problems were encountered
with this tool during the development stages.

This guide will present a setup procedure that supposedly works for all distributions based on the newest
LTS from Ubuntu. For other Linux distributions or operating system, please refer to the official documentation.

\subsubsection{Dependencies \& installation}
\label{sec:orga62ed7c}
Some packages are required to be installed on the host system
in order for the installation process to happen successfully:

\begin{minted}[frame=single,framesep=2mm,baselinestretch=1.2,linenos,breaklines,fontsize=\footnotesize]{sh}
sudo apt-get -y update

sudo apt-get -y install libncurses-dev \
     ncurses-term \
     ncurses-base \
     ncurses-bin \
     libncurses5 \
     libtinfo5 \
     libncurses5-dev \
     libncursesw5-dev
\end{minted}

Once this is done, the previously downloaded binary installer can be executed:

\begin{minted}[frame=single,framesep=2mm,baselinestretch=1.2,linenos,breaklines,fontsize=\footnotesize]{sh}
./Xilinx_Unified_2022.2_1014_8888_Lin64.bin
\end{minted}

If it is not possible to run the previous command, make the file executable with the \texttt{chmod} command:

\begin{minted}[frame=single,framesep=2mm,baselinestretch=1.2,linenos,breaklines,fontsize=\footnotesize]{sh}
sudo chmod +x ./Xilinx_Unified_2022.2_1014_8888_Lin64.bin
\end{minted}

From there you can follow the step-by-step graphical installer.
The directory chosen for the rest of this guide for the Xilinx directory
is directly the \texttt{\$HOME}, but the installation can be set elsewhere is needed.

\begin{tcolorbox}[colback=orange!5!white,colframe=orange!75!black]
\textbf{WARNING}: This whole procedure can take up to multiple hours to complete
and is prone to failures (regarding missing dependencies, typically),
so your schedule should be arranged accordingly.
\end{tcolorbox}

\subsubsection{Platform configuration file generation}
\label{sec:org7ad0300}
In order to have the libraries and configurations in the IDE ready to be used for our board,
we need to obtain some configuration files that are specific for the Kria KV260,
as presented in the \href{https://xilinx.github.io/kria-apps-docs/kv260/2022.1/build/html/docs/build\_vitis\_platform.html?highlight=xsa}{Xilinx guide for Kria and Vitis}.

A Xilix \href{https://github.com/Xilinx/kria-vitis-platforms}{dedicated repository} is available for us to download  such configurations,
but they required to be built.

As for the dependencies, \texttt{Cmake}, \texttt{tcl} and \texttt{idn} will become needed in order to build the firmware.
Regarding \texttt{idn}, some version issue can happen, but as discussed \href{https://support.xilinx.com/s/question/0D52E00006jrzsYSAQ/platform-project-cannot-be-created-on-vitis?language=en\\\_US}{in a thread on Xilinx's forum},
if \texttt{libidn11} is specifically required but not available (it is the case for Ubuntu 22.04),
creating a symbolic link from the current, 12 version works as a workaround.

Here are the steps for installing the dependencies and building this configuration file:

\begin{minted}[frame=single,framesep=2mm,baselinestretch=1.2,linenos,breaklines,fontsize=\footnotesize]{sh}
sudo apt-get update
sudo apt-get install cmake tcl libidn11-dev \
libidn-dev libidn12 idn
sudo ln -s /usr/lib/x86_64-linux-gnu/libidn.so.12 \
/usr/lib/x86_64-linux-gnu/libidn.so.11

cd ~/Xilinx
git clone --recursive \
https://github.com/Xilinx/kria-vitis-platforms.git
cd kria-vitis-platforms/k26/platforms
export XILINX_VIVADO=/home/$USER/Xilinx/Vivado/2022.2/
export XILINX_VITIS=/home/$USER/Xilinx/Vitis/2022.2/
make platform PLATFORM=k26_base_starter_kit
\end{minted}


\subsection{Setting up and building a new project for the Kria board}
\label{sec:orga174917}
With the platform configuration files available, we can now use the IDE to generate a
new project for our board. The whole process will be described with screen captures and
captions.

\begin{figure}[htbp]
\centering
\includegraphics[width=.5\textwidth]{./img/vitis_new/project1.png}
\caption{\label{fig:org18cec81}We are starting with creating a "New Application Project" You should be greeted with this wizard window. Next.}
\end{figure}

\begin{figure}[htbp]
\centering
\includegraphics[width=.5\textwidth]{./img/vitis_new/project2.png}
\caption{\label{fig:orgbb6ade9}For the platform, we need to get our build Kria configuration. In the "Create a new platform" tab, click the "Browse\ldots{}" button.}
\end{figure}

\begin{figure}[htbp]
\centering
\includegraphics[width=.6\textwidth]{./img/vitis_new/project3.png}
\caption{\label{fig:orgcfb9d9f}In the file explorer, we should navigate in the "k26" directory, where the configuration file was build. From here we are looking for a ".xsa" file, located in a "hw" directory, as visible.}
\end{figure}

\begin{figure}[htbp]
\centering
\includegraphics[width=.6\textwidth]{./img/vitis_new/project4.png}
\caption{\label{fig:org3f5c3be}With the configuration file loaded, we can now select a name for our platform, but most importantly, we have to select the "psu Cortex5 0" core as a target. The other, Cortex 53 is the APU running Linux.}
\end{figure}

\begin{figure}[htbp]
\centering
\includegraphics[width=.6\textwidth]{./img/vitis_new/project5.png}
\caption{\label{fig:org88f936d}In this next window, we can give a name to our firmware project. It is also critical here to select the core we want to build for. Once again, we want to use the "psu cortex5 0".}
\end{figure}

\begin{figure}[htbp]
\centering
\includegraphics[width=.6\textwidth]{./img/vitis_new/project6.png}
\caption{\label{fig:orgadbe7bd}Here, we want to select "freertos10 xilinx" as our Operating System. The rest can remain unchanged.}
\end{figure}

\begin{figure}[htbp]
\centering
\includegraphics[width=.6\textwidth]{./img/vitis_new/project7.png}
\caption{\label{fig:org04a5a70}Finally, we can select the demonstration template we are going to use; here we go with "OpenAMP echo-test" since we want to have some simple try of the RPMsg system. Finish.}
\end{figure}

\pagebreak
In the Xilinx documentation, it is made mention of the addresses setting that should be checked in the \texttt{script.ld} file.
These valued look different from what could be set in the DTO for the Linux side, but they appear to
work for the example we are running:

\begin{minted}[frame=single,framesep=2mm,baselinestretch=1.2,linenos,breaklines,fontsize=\footnotesize]{sh}
psu_ddr_S_AXI_BASEADDR                     0x3ed00000	
psu_ocm_ram_1_S_AXI_BASEADDR        0xfffc0000
psu_r5_tcm_ram_0_S_AXI_BASEADDR    0x00000000
psu_r5_tcm_ram_1_S_AXI_BASEADDR    0x00020000	
\end{minted}



\pagebreak
\section{Building a real-time firmware}
\label{sec:org8b58b57}

\subsection{Setting up Vitis IDE}
\label{sec:org505cbd8}

\section{RPMsg \texttt{echo\_test} software}
\label{sec:orgca792f3}
In order to test the deployment of the firmware on the R5F side, and in particular
to test the RPMsg function, we need some program on the Linux side of the Kria
board to "talk" with the real-time side.

Some source is provided by Xilinx to build a demonstration software that does
this purpose: specifically interact with the demonstration firmware.

Here are the steps required to obtain the sources, and build the program.

As a reminder, this is meant to be done on the Linux running on the
Kria board, NOT on your host machine !

\begin{minted}[frame=single,framesep=2mm,baselinestretch=1.2,linenos,breaklines,fontsize=\footnotesize]{sh}
git clone https://github.com/Xilinx/meta-openamp.git
cd  meta-openamp
git checkout xlnx-rel-v2022.2
cd  ./recipes-openamp/rpmsg-examples/rpmsg-echo-test
make
sudo ln -s $(pwd)/echo_test /usr/bin/
\end{minted}

Once this is done, it it possible to run the test program from the Kria board's Ubuntu
by running the \texttt{echo\_test} command.

\section{Building micro-ROS as a static library}
\label{sec:org6891b4e}


\begin{minted}[frame=single,framesep=2mm,baselinestretch=1.2,linenos,breaklines,fontsize=\footnotesize]{sh}
pushd /home/$USER/Downloads
wget https://developer.arm.com/-/media/Files/downloads/\
gnu/12.2.mpacbti-rel1/binrel/arm-gnu-toolchain-12.2\
.mpacbti-rel1-x86_64-arm-none-eabi.tar.xz
tar -xvf arm-gnu-toolchain-12.2.mpacbti-rel1-x86_64-\
arm-none-eabi.tar.xz
popd

toolchain="/home/$USER/Downloads/arm-gnu-toolchain-\
12.2.mpacbti-rel1-x86_64-arm-none-eabi/"


docker run -d --name ros_build -it --net=host \
--hostname ros_build \
-v /dev:/dev \
-v $toolchain:/armr5-toolchain \
--privileged ros:iron

docker exec -it ros_build bash
\end{minted}

\section{Adding micro-ROS to a firmware project}
\label{sec:org5b464de}

\section{Loading a real-time firmware}
\label{sec:orgd8f92b8}

\section{Running a ROS2 node}
\label{sec:org3e74ac3}

\subsection{On the host Linux}
\label{sec:org5595693}

\subsection{In a container}
\label{sec:orga452249}

\section{micro-ROS agent}
\label{sec:orga0f82bd}

\pagebreak
\appendix
\section{DTO patch}
\label{sec:orgf1db9a6}
This file is available in this repository: \href{https://gitlab.com/sunoc/xilinx-kria-kv260-documentation/-/blob/b7300116e153f4b5a1542f8804e4646db8030033/src/system.patch}{system.patch}
\inputminted[linenos, frame=single]{diff}{./src/system.patch}

\pagebreak
\section{Custom toolchain CMake settings}
\label{sec:org7640972}
This file is available in this repository: \href{https://gitlab.com/sunoc/xilinx-kria-kv260-documentation/-/blob/b7300116e153f4b5a1542f8804e4646db8030033/src/custom\_r5f\_toolchain.cmake}{custom r5f toolchain.cmake}
\inputminted[linenos, frame=single]{cmake}{./src/custom_r5f_toolchain.cmake}

\pagebreak
\section{Custom Colcon meta settings}
\label{sec:orgbbc3767}
This file is available in this repository: \href{https://gitlab.com/sunoc/xilinx-kria-kv260-documentation/-/blob/b7300116e153f4b5a1542f8804e4646db8030033/src/custom\_r5f\_colcon.meta}{custom r5f colcon.meta}
\inputminted[linenos, frame=single]{yaml}{./src/custom_r5f_colcon.meta}

\pagebreak
\section{Firmware time functions}
\label{sec:orgbd9cbf2}

\subsection{main}
\label{sec:org4422d2f}
This file is available in this repository: \href{https://gitlab.com/sunoc/xilinx-kria-kv260-documentation/-/blob/b7300116e153f4b5a1542f8804e4646db8030033/src/clock.c}{clock.c}
\inputminted[linenos, frame=single]{c}{./src/clock.c}

\subsection{header file}
\label{sec:org3ea11dc}
\begin{minted}[frame=single,framesep=2mm,baselinestretch=1.2,linenos,breaklines,fontsize=\footnotesize]{c}
/**< Microseconds per second. */
#define MICROSECONDS_PER_SECOND    ( 1000000LL )  
/**< Nanoseconds per second. */
#define NANOSECONDS_PER_SECOND     ( 1000000000LL ) 
/**< Nanoseconds per FreeRTOS tick. */  
#define NANOSECONDS_PER_TICK       ( NANOSECONDS_PER_SECOND / configTICK_RATE_HZ ) 
\end{minted}


\pagebreak
\section{Firmware memory allocation functions}
\label{sec:orgfa6dba0}

\subsection{main}
\label{sec:org37c4872}
This file is available in this repository: \href{https://gitlab.com/sunoc/xilinx-kria-kv260-documentation/-/blob/b7300116e153f4b5a1542f8804e4646db8030033/src/allocators.c}{allocators.c}
\inputminted[linenos, frame=single]{c}{./src/allocators.c}

\subsection{header file}
\label{sec:org572f81b}
\begin{minted}[frame=single,framesep=2mm,baselinestretch=1.2,linenos,breaklines,fontsize=\footnotesize]{c}
#ifndef _ALLOCATORS_H_
#define _ALLOCATORS_H_

#include "microros.h"

extern int absoluteUsedMemory;
extern int usedMemory;


void * __freertos_allocate(size_t size, void * state);
void __freertos_deallocate(void * pointer, void * state);
void * __freertos_reallocate(void * pointer, size_t size, void * state);
void * __freertos_zero_allocate(size_t number_of_elements,
size_t size_of_element, void * state);

#endif // _ALLOCATORS_H_
\end{minted}
\end{document}